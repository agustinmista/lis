\documentclass[a4paper,10pt]{article}
\usepackage[utf8]{inputenc}
\usepackage{fancyhdr, float, graphicx, caption}
\usepackage{amsmath, amssymb}
\usepackage{bm}
\usepackage[margin=1in]{geometry}
\usepackage{multicol}
\usepackage{proof}
\usepackage{titlesec} 

\setlength{\inferLineSkip}{4pt}

\titleformat{\subsection}[runin]
  {\normalfont\large\bfseries}{\thesubsection}{1em}{}	
\titleformat{\subsubsection}[runin]
  {\normalfont\normalsize\bfseries}{\thesubsubsection}{1em}{}


\pagestyle{fancy}
\renewcommand{\figurename}{Figura}
\renewcommand\abstractname{\textit{Abstract}}

\fancyhf{}
\fancyhead[LE,RO]{\textit{Lenguaje Interpretado Simple}}
\fancyfoot[RE,CO]{\thepage}

%%%%%%%%%%%%%%%%%%%%%%%%%%%%%%%%%%%%%%%%%%%%%%%%%%%%%%%%%%%%

\title{
	%Logo UNR
	\begin{figure}[!h]
		\centering
		\includegraphics[scale=1]{unr.png}
		\label{}
	\end{figure}
	% Pie Logo
	\normalsize
		\textsc{Universidad Nacional de Rosario}\\	
		\textsc{Facultad de Ciencias Exactas, Ingeniería y Agrimensura}\\
		\textit{Licenciatura en Ciencias de la Computación}\\
		\textit{Estructuras de datos y algoritmos II}\\
	% Título
	\vspace{30pt}
	\hrule{}
	\vspace{15pt}
	\huge
		\textbf{Lenguaje Interpretado Simple}\\
	\vspace{15pt}
	\hrule{}
	\vspace{30pt}
	% Alumnos/docentes
	\begin{multicols}{2}
	\raggedright
		\large
			\textbf{Alumnos:}\\
		\normalsize
			CRESPO, Lisandro (C-6165/4) \\
			MISTA, Agustín (M-6105/1) \\
			.\\
			.\\
	\raggedleft
		\large
			\textbf{Docentes:}\\
		\normalsize
			JASKELIOFF, Mauro\\
			RABASEDAS, Juan Manuel\\
			SIMICH, Eugenia\\
			MANZINO, Cecilia\\
	\end{multicols}
}
%%%%%%%%%%%%%%%%%%%%%%%%%%%%%%%%%%%%%%%%%%%%%%%%%%%%%%%%%%%
\begin{document}
\date{28 de Agosto de 2015}
\maketitle

\pagebreak
%----------------------------------------------------------
\subsection*{Ejercicio 2.2.1.} 
	\emph{Extendemos la sintaxis abstracta y concreta de las expresiones enteras del LIS
	a modo de incluir el operador de asignación ternario del lenguaje C.}
	\\
	\\
	\textbf{Sintaxis abstracta}
	\begin{alignat*}{3}
		\langle intexp \rangle\ ::&= \langle nat \rangle\ |\ \langle &&var \rangle\ &&|\ {-u}\ \langle intexp \rangle \\
		&|\;\langle intexp  \rangle  &&\;{+}      &&\langle intexp \rangle\\
		&|\;\langle intexp  \rangle  &&\;{-b}     &&\langle intexp \rangle\\
		&|\;\langle intexp  \rangle  &&\;{\times} &&\langle intexp \rangle\\
		&|\;\langle intexp  \rangle  &&\;{\div}   &&\langle intexp \rangle\\
		&|\;\langle boolexp \rangle  &&\;{?}      &&\langle intexp \rangle \;:\; \langle intexp \rangle \\
	\end{alignat*}
	\textbf{Sintaxis concreta}
	\begin{alignat*}{4}
		\langle intexp \rangle\ ::&= \;&&\langle nat \rangle\\
		&| &&\langle var \rangle \\
		&| &&\bm{`-`}\;\langle intexp \rangle\\
		&| &&\langle intexp \rangle \; \bm{`+`}\; \langle intexp \rangle\\
		&| &&\langle intexp \rangle \; \bm{`-`}\; \langle intexp \rangle\\
		&| &&\langle intexp \rangle \; \bm{`*`}\; \langle intexp \rangle\\
		&| &&\langle intexp \rangle \; \bm{`/`}\; \langle intexp \rangle\\
		&| &&\bm{`(`}\;\langle intexp \rangle\;\bm{`)`}\\
		&| &&\langle boolexp \rangle\;\bm{`?`}\;\langle intexp \rangle\;\bm{`:`}\; \langle intexp \rangle \\
	\end{alignat*}
\\
%----------------------------------------------------------
\subsection*{Ejercicio 2.3.1.}
	\emph{Extendemos la sintaxis abstracta de las expresiones enteras en Haskell para incluir el operador de asignación ternario descripto en el Ejercicio 2.2.1.}
	\\
	\begin{alignat*}{3}
	data IntExp=&\;Const\;     &&Int\\
	    		&|\;Var\;      &&Variable\\
				&|\;UMinus\;\; &&IntExp\\
				&|\;Plus\;     &&IntExp\;     &&IntExp\\
				&|\;Minus\;    &&IntExp\;     &&IntExp\\
				&|\;Times\;    &&IntExp\;     &&IntExp\\
				&|\;Div\;      &&IntExp\;     &&IntExp\\
				&|\;IfAss\;    &&BoolExp\;\;  &&IntExp\;\; IntExp\\
	\end{alignat*}
\\
\subsection*{Ejercicio 2.4.1.}
	\emph{Extendemos la semántica denotacional de las expresiones enteras para incluir el operador ternario descripto en el Ejercicio 2.2.1}
	\\
	\begin{align*}
		[\![ cond\;?\; expT\;:\;expF ]\!]_{intexp}\sigma = 
		\begin{cases} 
			[\![expT]\!]_{intexp}\sigma &\mbox{si } [\![cond]\!]_{boolexp}\sigma = \textbf{true} \\
			[\![expF]\!]_{intexp}\sigma &\mbox{si } [\![cond]\!]_{boolexp}\sigma = \textbf{false} \\
		\end{cases}
	\end{align*}
\pagebreak
%----------------------------------------------------------
\subsection*{Ejercicio 2.5.1.}
	\emph{Demostrar que la relación de evaluación de un paso $\rightsquigarrow$ es determinista}
	\\
	\begin{align*}
		Blah,\;blah,\;blah..
	\end{align*}
\\
%----------------------------------------------------------
\subsection*{Ejercicio 2.5.2.}
	\emph{Construimos un árbol de prueba para demostrar que:}
	\\
	\begin{align*}
		\langle x:= x+1;\;{\bf if}\; x>0\;{\bf then\;skip\;else}\; x:=x-1, [\sigma|x:0]\rangle \rightsquigarrow^*[\sigma|x:1]
	\end{align*}
	\\
	% raíz
	\begin{align*}
		\infer[TR2]{
			\langle x:= x+1;\;{\bf if}\; x>0\;{\bf then\;skip\;else}\; x:=x-1, [\sigma|x:0]\rangle \rightsquigarrow^*[\sigma|x:1]
		}{
			(1)\qquad \qquad \qquad & \qquad \qquad \qquad(2)
		}
	\end{align*}
	% eq (1)
	\begin{align}
		\infer[TR1]{
			\langle x:= x+1;\;{\bf if}\; x>0\;{\bf then\;skip\;else}\; x:=x-1, [\sigma|x:0]\rangle \rightsquigarrow^* \langle {\bf if}\; x>0\;{\bf then\;skip\;else}\; x:=x-1, [\sigma|x:1]\rangle
		}{
			\infer[SEQ1]{
				\langle x:= x+1;\;{\bf if}\; x>0\;{\bf then\;skip\;else}\; x:=x-1, [\sigma|x:0]\rangle \rightsquigarrow \langle {\bf if}\; x>0\;{\bf then\;skip\;else}\; x:=x-1, [\sigma|x:1]\rangle
			}{
				\infer[ASS^{(3)}]{
					\langle x:= x+1, [\sigma|x:0] \rangle \rightsquigarrow [\sigma|x:1]
				}{
				}
			}
		}
	\end{align}
	% eq (2)
	\begin{align}
		\infer[TR2]{
			\langle {\bf if}\; x>0\;{\bf then\;skip\;else}\; x:=x-1, [\sigma|x:1]\rangle \rightsquigarrow^* [\sigma|x:1]
		}{
			\infer[TR1]{
				\langle {\bf if}\; x>0\;{\bf then\;skip\;else}\; x:=x-1, [\sigma|x:1]\rangle \rightsquigarrow^* \langle {\bf skip}, [\sigma|x:1]\rangle
			}{
				\infer[IF1]{
					\langle {\bf if}\; x>0\;{\bf then\;skip\;else}\; x:=x-1, [\sigma|x:1]\rangle \rightsquigarrow \langle {\bf skip}, [\sigma|x:1]\rangle
				}{
					[\![x>0]\!][\sigma|x:1] = {\bf true}^{(4)}
				}
			} & \qquad
			\infer[TR1]{
				\langle {\bf skip}, [\sigma|x:1]\rangle \rightsquigarrow^* [\sigma|x:1]
			}{
				\infer[SKIP]{
					\langle {\bf skip}, [\sigma|x:1]\rangle \rightsquigarrow [\sigma|x:1]
				}{
				}
			}
		}
	\end{align}
	\\
	\\
	% pruebas de semántica
	\emph{Además, probamos que:}
	\begin{alignat}{3}
		[\![x+1]\!][\sigma|x:0] \quad &= \quad [\![x]\!][\sigma|x:0] + [\![1]\!][\sigma|x:1] \quad &&= \quad 0+1 \quad &&= \quad {\bf 1}\\
	  	[\![x>0]\!][\sigma|x:1] \quad &= \quad [\![x]\!][\sigma|x:1] > [\![0]\!] \quad &&= \quad 1>0 \quad &&= \quad {\bf true}
	\end{alignat}
\\
%----------------------------------------------------------
\subsection*{Ejercicio 2.5.6.}
	\emph{Agregamos una producción a la gramática abstracta de los comandos del LIS para el comando {\bf repeat}}
	\\
	\begin{alignat*}{3}
		\langle comm \rangle\ ::&= {\bf skip} \\
			&|\;\langle var  \rangle  \;{:=}\;\langle intexp \rangle\\
			&|\;\langle comm \rangle  \;{;}\;\langle comm \rangle\\
			&|\;{\bf if}\;\langle boolexp \rangle\;{\bf then}\;\langle comm \rangle\;{\bf else}\;\langle comm \rangle\\
			&|\;{\bf while}\;\langle boolexp \rangle\;{\bf do}\;\langle comm \rangle\\
			&|\;{\bf repeat}\;\langle comm \rangle\;{\bf until}\;\langle boolexp \rangle
	\end{alignat*}
	\\
	\emph{Extendemos la semántica operacional del LIS con reglas de inferencia para el comando {\bf repeat}}
	\\
	\begin{align*}
		\infer[REP1]{
			\langle {\bf repeat}\; c\;{\bf until}\; b,\;\sigma\rangle \rightsquigarrow \langle c; {\bf repeat}\; c\;{\bf until}\; b,\;\sigma'\rangle
		}{
			\langle c, \sigma\rangle \rightsquigarrow \sigma'
			\qquad & \qquad 
			[\![b]\!]_{boolexp}\sigma' = {\bf false} 
		}
	\end{align*}
	\\
	\begin{align*}
		\infer[REP2]{
			\langle {\bf repeat}\; c\;{\bf until}\; b,\;\sigma\rangle \rightsquigarrow \sigma'
		}{
			\langle c, \sigma\rangle \rightsquigarrow \sigma'
			\quad & \quad  
			[\![b]\!]_{boolexp}\sigma' = {\bf true} 
		}
	\end{align*}
	\\ 
\pagebreak
\\
\vspace{\fill}
\begin{multicols}{2}
	\hrule
	\vspace{5pt}
	CRESPO, Lisandro \\
	\linebreak

	\hrule
	\vspace{5pt}
	MISTA, Agustín \\
\end{multicols}

\end{document}
